%==============================================================================
%== template for LATEX poster =================================================
%==============================================================================
%
%--A0 beamer slide-------------------------------------------------------------
\documentclass[final]{beamer}
\usepackage[orientation=portrait,size=a0,
            scale=1.55        % font scale factor
           ]{beamerposter}
           
\geometry{
  hmargin=2.5cm, % little modification of margins
}

%
\usepackage[utf8]{inputenc}

\linespread{1.15}
%
%==The poster style============================================================
\usetheme{sharelatex}

%==Title, date and authors of the poster=======================================
\title
[XVI Semana da Estatística da UEM 5 - 7 Dezembro de 2022] % Conference
{ % Poster title
WeatherMaringa
}

\author{ % Authors
Lucas Stefano Xavier de Sousa\inst{1}, Brian Alvarez Ribeiro de Melo\inst{1}
}
\institute
[Universidade Estadual de Maringá] % General University
{
\inst{1} Departamento de Estatística - Universidade Estadual de Maringá\\[0.3ex]
}



\begin{document}
\begin{frame}[t]
%==============================================================================
\begin{multicols}{2}
%==============================================================================
%==The poster content==========================================================
%==============================================================================

\section{Introdução}

O monitoramento do clima sempre foi essencial para as atividades do homem. Nos últimos
tempos, com o avanço da tecnologia, uma grande quantidade de dados é gerada, principalmente no que
diz respeito às condições climáticas. Esses dados podem auxiliar na tomada de decisões estratégicas e
preventivas de um local. Contudo a existência e obtenção dessas informações nem sempre é conhecida.
Em Maringá, esses dados são registrados e compilados pela Estação Climatológica Principal de Maringá
(ECPM), que pertence à rede de estações do Instituto Nacional de Meteorologia (INMET), através dos
horários-padrão estabelecidos pela Organização Meteorológica Mundial (OMM): 12, 18 e 24 do Tempo
Médio de Greenwich (TMG), que correspondem às 9h, 15h e 21h do horário de Brasília.

Esses dados incluem temperatura do ar, velocidade e direção do vento, umidade relativa do ar, radiação solar, nebulosidade, precipitação, pressão atmosférica, entre outros. Após o registro e compilação dos dados meteorológicos, a ECPM disponibiliza seu banco de dados meteorológicos composto por boletins, que contém uma série histórica completa, permitindo analisar o tempo e o clima de Maringá.

\section{Objetivo}

O objetivo geral do projeto consiste no desenvolvimento e disponibilização de um pacote do ambiente de programação R que permita a manipulação de dados meteorológicos de Maringá. De forma especifica o pacote possui três objetivos principais:
\begin{itemize}
  \item Facilitar a pesquisa e coleta de dados meteorológicos de Maringá;
  \item Criar relatórios das séries históricas e atuais dos dados meteorológicos;
  \item Analisar e compreender as mudanças climáticas que ocorrem em Maringá ao longo do tempo.
\end{itemize}


\begin{figure}
\centering
\includegraphics[width=0.40\columnwidth]{hexlogo.png}
\caption{Logo hexagonal do pacote com as cores da bandeira de Maringá}
\end{figure}




\section{Metodologia}

Na execução deste projeto, seguimos os seguintes passos: em primeiro lugar, a obtenção do banco de  dados em formato CSV disponibilizado pela ECPM e pelo INMET. Para esta tarefa, utilizamos os pacotes do R:  \textit{Plumber} que permite criar uma API para automatizar as requisições de novos dados e   \textit{Tidyverse}  que inclui os diversos pacotes para análise e visualização de dados. Em seguida, utilizando o pacote  \textit{devtools}, que possui funções para criar pacotes e testar as funções, realizamos a construção, checagem e documentação do pacote dentro da nossa sessão do R. Durante o tempo de desenvolvimento, as tarefas repetitivas serão automatizadas utilizando o pacote  \textit{usethis} e o projeto será hospedado no Github e seu desenvolvimento ocorre na branch “dev”. Em terceiro lugar, dentro da própria branch “dev”, utilizamos o pacote  \textit{pkgdown} para tornar fácil e rápida a construção do site para o pacote. Nesta etapa, inciamos a documentação do pacote explicando suas funcionalidades e formas de uso. Em quarto lugar, utilizando o pacote  \textit{hexSticker} será desenvolvido o  \textit{sticker} hexágono do pacote. E por fim, o projeto será oficializado na branch  \textit{main} do repositório do Github para que seja possível a utilização do pacote em produção e manutenção. 
Após o pacote estar finalizado na branch  \textit{main} do Github, foi realizada a  submissão do pacote na  \textit{CRAN (The Comprehensive R Archive Network)}, enviando o pacote em formato tar.gz juntamente com o  \textit{DESCRIPTION file} do projeto. O pacote ainda está em processo de validação após isso, poderá ser baixado direto da CRAN através do comando  \textit{install.packages(“weathermaringa”)}.


\section{Resultados}

Como resultado do desenvolvimento o pacote possui 3 funções básicas que já estão em produção, essas funções são responsáveis pelo funcionamento central do pacote e foram validas nos testes de performance do fluxo do  Github Actons:
\begin{itemize}
  \item weathermga-stats(): Apresenta uma lista de subfunções de análise descritiva utilizando o R Base;
  \item weathermga-dl(): Download dos dados meteorológicos de Maringá;
  \item weathermga-interp(): Interpola e adiciona os dados meteorológicos a um dataframe;
  \item weathermga-temp(): Gera uma série histórica dos últimos 5 dias de Maringá.
\end{itemize}


\section{Discussão e Conclusão}

Durante o desenvolvimento do pacote foi possível definir de forma principal que a obtenção de dados meteorológicos e climáticos de Maringá passou a ficar mais rápida e fácil, pois com apenas uma função o download das bases do INMET ou ECPM é realizado dentro do próprio R, já ficando anexada as funções de análise descritiva do pacote, este processo diminui o tempo de carregamento a cada chamada de função.


%==============================================================================
%==End of content==============================================================
%==============================================================================

%--References------------------------------------------------------------------

\subsection{Referências}

\begin{thebibliography}{99}

\bibitem{ref2} Hamilton, J. M., Maddison, D. J., & Tol, R. S. (2005). \textit{Climate change and international tourism: a simulation study}. Global environmental change, 15(3), 253-266.

\bibitem{ref3} R Core Team (2020). \textit{R: A language and environment for statistical computing}. R Foundation for Statistical Computing, Vienna, Austria.

\end{thebibliography}
%--End of references-----------------------------------------------------------

\end{multicols}

%==============================================================================
\end{frame}
\end{document}
